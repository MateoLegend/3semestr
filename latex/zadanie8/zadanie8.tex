\documentclass{article}

\usepackage{amsmath}
\usepackage{amssymb}
\usepackage{latexsym}

\usepackage[utf8]{inputenc}
\usepackage{polski}

\usepackage[margin=2cm]{geometry}

\newcommand{\wydzial}{\textbf{Wydział Budownictwa, Mechaniki i Petrochemii}}

\newcommand{\kierunek}[1]{\textbf{\textsf{ #1 }}}

\newcommand{\student}[3]{%
Imię i nazwisko: \textbf{#1 #2}	\\
Liczba punktów ECTS: \textsf{#3}\\	
}

\begin{document}
	\noindent
	{\tiny tekst normalny}			\\
	{\scriptsize tekst normalny}	\\
	{\footnotesize tekst normalny}	\\
	{\small tekst normalny}			\\
	tekst normalny					\\
	{\large tekst normalny}			\\
	{\Large tekst normalny}			\\
	{\LARGE tekst normalny}			\\
	{\huge tekst normalny}			\\
	{\Huge tekst normalny}			\\
	
	\noindent
	\textbf{tekst pogrubiony}		\\
	\textsf{tekst bezszeryfowy}		\\
	\textit{tekst italic}			\\
	\textsl{tekst pochylony}		\\
	\texttt{tekst maszynowy}		\\
	\underline{tekst podkreślony}	\\
	
	\begin{center}
		{\Large \textbf{Tekst powiększony pogrubiony}}
	\end{center}
	\begin{center}
		{\textbf{\Large Tekst powiększony pogrubiony}}
	\end{center}
	\begin{center}
		{\Large \textbf{\textsf{Tekst powiększony pogrubiony}}}
	\end{center}
	
	\noindent
	\wydzial	\\
	\kierunek{Przemysłowe Zastosowania Informatyki}	\\
	\kierunek{Ekonomia}	\\
	\student{Janusz}{Kowalski}{34}\\
	\student{Anna}{Marzec}{31}\\
	
\end{document}

