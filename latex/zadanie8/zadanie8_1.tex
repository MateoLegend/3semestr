\documentclass{article}

\usepackage{amsmath}
\usepackage{amssymb}
\usepackage{latexsym}

\usepackage[utf8]{inputenc}
\usepackage{polski}

\usepackage[margin=2cm]{geometry}

\usepackage{xcolor}
\usepackage{tabularray}

\begin{document}
	
	\begin{tabular}{lccr}
		\hline
		Alpha & Beta & Gamma & Delta \\
		\hline
		Epsilon & Zeta & Eta & Theta \\
		\hline
		Iota & Kappa & Lambda & Mu \\
		\hline
	\end{tabular}
	
	\begin{tblr}{colspec={Q[l,brown7]Q[c,yellow7]Q[r,olive7]},rowspec={|Q|Q|Q|}}
		Alpha & Beta & Gamma \\
		Epsilon & Zeta & Eta \\
		Iota & Kappa & Lambda \\
	\end{tblr}
	
	
	\begin{tblr}{hlines, vlines}
		Poniedziałek & Miernictwo i systemy pomiarowe & Aparatura Przemysłowa & Podstawy inżynierii mechanicznej(lab)	\\
		Wtorek & Podstawy inżynierii mechanicznej (wyk) & Systemy operacyjne (wyk) & Systemy operacyjne (lab) & Cyfrowa dokumentacja techniczna	\\
		Środa & Język angielski B2 & Programowanie obiektowe (wyk) & Programowanie obiektowe (lab)	\\
		Czwartek & Relacyjne bazy danych (lab) & Relacyjne bazy danych (wyk) & Język angielski B2	\\
		Piątek
		
	\end{tblr}
	
	
\end{document}